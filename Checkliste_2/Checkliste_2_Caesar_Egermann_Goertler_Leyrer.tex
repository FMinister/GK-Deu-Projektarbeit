\documentclass[a4paper,
    11pt,
    headings=small,
    ngerman,
    listof=totoc,
    numbers=noenddot]{scrreprt}[2021/11/13]
\usepackage{ifxetex,ifluatex}
\ifcase \ifxetex 1\else\ifluatex 1\else 0\fi\fi\usepackage[utf8]{inputenc}\fi
\usepackage[T1]{fontenc}
\usepackage[ngerman]{babel}
\usepackage[utf8]{inputenc}
\usepackage{setspace}
\onehalfspacing
\usepackage{lmodern}
\usepackage{csquotes}

\usepackage[automark,headsepline=.4pt]{scrlayer-scrpage}
\clearmainofpairofpagestyles
\pagestyle{scrheadings}
\renewcommand*{\chaptermarkformat}{%
	\chaptername~\thechapter\autodot\enskip}
\ihead{\headmark}
\setkomafont{pageheadfoot}{}
\renewcommand*{\sectionmarkformat}{}
\cfoot{\pagemark}
\footskip1cm

\usepackage[left=2.5cm,right=2.5cm,top=3cm,bottom=2.5cm]{geometry}

\usepackage{xcolor, soul}
\definecolor{codebackground}{rgb}{0.95, 0.95, 0.92}
\definecolor{Black}{rgb}{0, 0, 0}

%%% Für Abkürzungen und Abkürzungsverzeichnis %%%
\usepackage[printonlyused,withpage]{acronym}

%%% Für Quotes %%%
\usepackage{url}
\usepackage[ngerman]{varioref}
\usepackage{mwe}
\usepackage{hyperref}% Weil es so in der Frage enthalten war.
 \hypersetup{%draft, 								% no hyperlinking at all (useful in b/w printouts)
    colorlinks=true, breaklinks=true,
    urlcolor=Black, linkcolor=Black, citecolor=Black,
    linktoc=page, %
    bookmarksnumbered, bookmarksopenlevel=1, bookmarksdepth = section,%
    pdfstartview=FitV,
    }
\setlength{\parindent}{0em}
\usepackage{bookmark}% Weil das hyperref deutlich verbesser.
\usepackage{cleveref}
\crefname{paragraph}{Abschnitt}{Abschnitt}

%%% Für Textübergreifende Numerierung %%%
\usepackage{enumitem}
\renewcommand{\labelenumi}{\alph{enumi})}

%%% Um PDFs einzubinden %%%
\usepackage{pdfpages}

%%% Um Zahlen mit Einheiten korrekt darstellen %%%
\usepackage{siunitx}
\sisetup{
  locale = DE ,
  detect-all,
  binary-units = true
}

%%% Code schoen darstellen %%%
\usepackage{listings}
\lstdefinestyle{MyPythonStyle}{
  frame=tb, % hrule above and below
  keepspaces=true,
  breaklines=true,
  columns=flexible,
  basicstyle=\texttt\scriptsize,
  escapeinside={(*@}{@*)}, % for escaping
  backgroundcolor=\color{codebackground},
  showstringspaces=false,
  language=Python,
  keywordstyle=\color{blue},
  stringstyle=\color{red},
  commentstyle=\color{teal},
  numbers=left, % {none, left, right}
  firstnumber=1,
  numberstyle=\scriptsize\color{black},
  numbersep=5pt,
  xleftmargin=5.0ex,
  gobble=-4
}

%%% Korrekte darstellung fuer Sonderzeichen im Code
\lstset{literate=%
    {Ö}{{\"O}}1
    {Ä}{{\"A}}1
    {Ü}{{\"U}}1
    {ß}{{\ss}}1
    {ü}{{\"u}}1
    {ä}{{\"a}}1
    {ö}{{\"o}}1
    {~}{{\textasciitilde}}1
}


%%% Verzeichnisse im Anhang %%%
\DeclareNewTOC[%
  owner=\jobname,
  listname={Inhalt des Anhangs},% Titel des Verzeichnisses
]{atoc}% Dateierweiterung (a=appendix, toc=table of contents)
\DeclareNewTOC[%
  listname={Abbildungen im Anhang},% Titel des Verzeichnisses
]{alof}% Dateierweiterung (a=appendix, lof=list of figures)
\DeclareNewTOC[%
  listname={Tabellen im Anhang},% Titel des Verzeichnisses
]{alot}% Dateierweiterung (a=appendix, lot=list of tables)
 
\makeatletter
\newcommand*{\useappendixtocs}{%
  \renewcommand*{\ext@toc}{atoc}%
  \scr@ifundefinedorrelax{hypersetup}{}{% damit es auch ohne hyperref funktioniert
    \hypersetup{bookmarkstype=atoc}%
  }%
  \renewcommand*{\ext@figure}{alof}%
  \renewcommand*{\ext@table}{alot}%
}
\newcommand*{\usestandardtocs}{%
  \renewcommand*{\ext@toc}{toc}%
  \scr@ifundefinedorrelax{hypersetup}{}{% damit es auch ohne hyperref funktioniert
    \hypersetup{bookmarkstype=toc}%
  }%
  \renewcommand*{\ext@figure}{lof}%
  \renewcommand*{\ext@table}{lot}%
}
\scr@ifundefinedorrelax{ext@toc}{%
  \newcommand*{\ext@toc}{toc}
  \renewcommand{\addtocentrydefault}[3]{%
    \expandafter\tocbasic@addxcontentsline\expandafter{\ext@toc}{#1}{#2}{#3}%
  }
}{}
\makeatother
 
\usepackage{xpatch}
\xapptocmd\appendix{%
  \addpart{\appendixname}
  \useappendixtocs
}{}{}

%%% Alles bzgl. des Literaturverzeichnisses
\usepackage[bibencoding=utf8,
			sortlocale=de,
			style=numeric,
			pagetracker=true,
			autocite=inline,
			backrefstyle=three+,
			date=short,
			sorting=nty,
			backend=biber]{biblatex}
\bibliography{Literaturverzeichnis}

%%% urldate in eckigen Klammern %%%
\DeclareFieldFormat{urldate}{\mkbibbrackets{#1}}
%%% URL: = Verfügbar unter: %%%
\DeclareFieldFormat{url}{{Verfügbar unter:}\space\url{#1}}
%%% Abstand zwischen den Literaturangaben %%%
\setlength{\bibitemsep}{1.3em}
%%% statt und ein & %%%
\renewcommand*{\finalnamedelim}{\space\&\space}
%%% Nachname, Vorname, immer %%%
\DeclareNameAlias{sortname}{last-first}

\title{Die hard- und softwaretechnische Implementierung eines CO2-Sensors zur Messung der Raumluftqualität}
\author{Julius Caesar, Péter Egermann, Paul Görtler, Johannes Leyrer}
\date{02.03.2022}

\begin{document}
\maketitle


\chapter{Projektorganisation}

\begin{description}
  \item[Aufgabenverteilung] \
        \begin{itemize}
          \item \textbf{Wer übernimmt welche Aufgabe?}
                \begin{itemize}
                  \item Johannes: Bereitstellung Hardware und Software
                  \item Julius: Infos Raspberry Pi und Nutzung als Wifi-Hotspot
                  \item Paul: Project Owner, Docker, Dokumentation
                  \item Peter: Projektmanagement, CO2-Grenzwerte und ihre Auswirkungen auf den Menschen
                  \item Alle: Beurteilung und Verbesserung Software, gegenseitiges Kontrollieren der Einzelergebnisse
                \end{itemize}
          \item \textbf{Wie koordinieren wir unsere Arbeit? (Treffpunkte, Austausch)}

                Kommunikation über eine Signal-Gruppe, Online-Meetings via Discord, GitHub zur Versionsverwaltung und als Daten-Backup

          \item \textbf{Wie wird der Arbeitsprozess protokolliert und dokumentiert?}

                Agiles Vorgehensmodell mit GitHub, Dokumentation mittels LaTeX und Markup
        \end{itemize}
  \item[Zeitplan] \
        \begin{itemize}
          \item \textbf{Wie sieht ein grober Zeitplan aus? (Fixpunkte)}

                Wird mittels Ganttdiagramm gelöst, spätestens bis 06.05.2022

          \item \textbf{Wie kontrollieren wir die Einhaltung des Zeitplans?}

                Benachrichtigungen der Gantt-Diagramm-Anwendung und manuelle Überprüfung der Termine und Einzelfortschritte durch den Projektmanager
        \end{itemize}
  \item[Arbeitsort] \
        \begin{itemize}
          \item \textbf{Wo findet die Hauptarbeit der Gruppe statt?}

                In der Schule während der vorgesehenen Arbeitszeiten und online, wenn nötig.

          \item \textbf{Benötigt die Gruppe einen speziellen Arbeitsort?}

                Nein, ein einfacher Computerarbeitsplatz reicht für alle Aktivitäten.

          \item \textbf{Platzbedarf?}

                Es ist kein besonderer Platzbedarf nötig.
        \end{itemize}

        \newpage

  \item[Hilfsmittel/Material] \
        \begin{itemize}
          \item \textbf{Welche Hilfsmittel benötigt die Gruppe?}

                Raspberry Pi, CO2-Monitor, Netzstecker, USB-Kabel (A auf Micro)

          \item \textbf{Wo können diese beschafft werden?}

                Sind bereits komplett vorhanden.
        \end{itemize}
  \item[Arbeitsstil] \
        \begin{itemize}
          \item \textbf{Welche Schwierigkeiten/Probleme können auftreten?}

                Das einzige Problem könnte das Zeitmanagement werden. Das komplette Setup mit CO2-Sensor und Raspberry Pi ist bereits in der Praxis umgesetzt worden und stellt deswegen keine große Herausforderung dar.

          \item \textbf{Wie reagieren wir darauf?}

                Der Projektmanager behält die Abgabetermine im Blick und weist das Team im Vornherein auf knappe Deadlines oder etwaige Risiken hin.

          \item \textbf{Welche Gesprächs- und Entscheidungsregeln sollen in der Gruppe gelten?}

                Demokratisches Abstimmverhalten mit Veto-Recht für Project Owner. Ansonsten wird auf respektvolles und vertrauliches Miteinander gesetzt.
        \end{itemize}
\end{description}



\end{document}